\documentclass[10pt,letterpaper]{article}

\usepackage{hyperref}
\usepackage{geometry}

% Fonts
\usepackage[T1]{fontenc}
\usepackage[bitstream-charter]{mathdesign}
\usepackage[scaled=0.85]{beramono}

% Set your name here
\def\name{Song Cao}

% The following metadata will show up in the PDF properties
\hypersetup{
  colorlinks = true,
  urlcolor = black,
  pdfauthor = {\name},
  pdfkeywords = {computer science, computer software},
  pdftitle = {\name: Curriculum Vitae},
  pdfsubject = {Curriculum Vitae},
  pdfpagemode = UseNone
}

\geometry{
  body={6.5in, 9.0in},
  left=1.0in,
  top=1.0in
}

% Customize page headers
\pagestyle{myheadings} \markright{\name} \thispagestyle{empty}

% Custom section fonts
\usepackage{sectsty}
\sectionfont{\rmfamily\mdseries\Large}
\subsectionfont{\rmfamily\mdseries\itshape\large}

% Other possible font commands include:
% \ttfamily for teletype,
% \sffamily for sans serif,
% \bfseries for bold,
% \scshape for small caps,
% \normalsize, \large, \Large, \LARGE sizes.

% Don't indent paragraphs.
\setlength\parindent{0em}

% Make lists without bullets and compact spacing
\renewenvironment{itemize}{
  \begin{list}{}{
    \setlength{\leftmargin}{1.5em}
    \setlength{\itemsep}{0.25em}
    \setlength{\parskip}{0pt}
    \setlength{\parsep}{0.25em}
  }
}{
  \end{list}
}

\begin{document}

% Place name at left
{\huge \name}

% Alternatively, print name centered and bold:
%\centerline{\huge \bf \name}

\vspace{0.25in}

\begin{minipage}[t]{0.5\textwidth}
  Department of Computer Science \\
  Cornell University \\
  4144 Upson Hall \\
  Ithaca, NY 14850
\end{minipage}
\begin{minipage}[t]{0.5\textwidth}
  Mobile: (607) 220-6745 \\
  Email: \textit{caosong@cs.cornell.edu} \\
  Homepage: \textit{http://www.cs.cornell.edu/~caosong/} \\
\end{minipage}

\section*{Current Position}
5th Year PhD Candidate (Research Assistant), Department of Computer
Science, Cornell University

\section*{Previous Education}
\begin{itemize}
    \item B.E. Computer Software, Tsinghua University, Beijing, China, 2008.

    \begin{itemize}
    \item \textit{GPA:}
      92.1/100
%    \item \textit{Ranking:}
%      2/52
    \item \textit{Undergraduate Thesis:}
      Multiple Motion Trajectory-Based Video Retrieval
    \end{itemize}
\end{itemize}

\section*{Research Interests}
\begin{itemize}
\item Computer Vision and Applied Machine Learning (Advisor: Noah Snavely)
\end{itemize}

\section*{Publications}
\begin{itemize}
\item (In submission) Song Cao, Noah Snavely: Minimal Scene Descriptions from Structure from Motion Models. CVPR 2014.
\item Song Cao, Noah Snavely: Graph-Based Discriminative Learning for Location Recognition. CVPR 2013.
\item Song Cao, Noah Snavely: Learning to Match Images in Large-Scale Collections. Workshop on Web-scale Vision and Social Media, ECCV 2012.
\item \href{http://dx.doi.org/10.1016/j.cad.2007.10.011}{Yijun Yang, Song Cao, Junhai Yong, Hui Zhang, Jean-Claude Paul, Jiaguang Sun: Approximate computation of curves on B-spline surfaces. Computer-Aided Design, Volume 40, Issue 2, February 2008, Pages 223-234}.
%\item \href{http://dx.doi.org/10.1007/978-3-540-75867-9_153}{Cunhao Fang, Song Cao: A Practical Agent-Based Approach for Pattern Layout Design, Computer Aided Systems Theory EUROCAST 2007, Springer LNCS, Volume 4739/2007, Pages 1223-1228}.
%\item \href{http://dx.doi.org/10.1007/978-3-540-75867-9_61}{Cunhao Fang, Yaoxue Zhang, Song Cao: A Resources Virtualization Approach Supporting Uniform Access to Heterogeneous Grid Resources, Computer Aided Systems Theory EUROCAST 2007, Springer LNCS, Volume 4739/2007, Pages 481-487}.
\end{itemize}

\section*{Selected Projects}
\subsection*{Minimal Scene Descriptions from Structure from Motion Models}
\begin{itemize}
\item January 2013 - November 2013
\item How much data do we need to describe a location? We explore this question in the context of 3D scene reconstructions created from running structure from motion on large Internet photo collections, where reconstructions can contain many millions of 3D points. We consider several methods for computing much more compact representations of such reconstructions for the task of location recognition, with the goal of maintaining good performance with very small models. In particular, we introduce a new method for computing compact models that takes into account both image-point relationships, as well as feature distinctiveness, and show that this method produces small models that yield better recognition performance than previous model reduction techniques.
\end{itemize}

\subsection*{Graph-Based Discriminative Learning for Location Recognition}
\begin{itemize}
\item January 2012 - November 2012
\item Recognizing the location of a query image by matching it to a database is an important problem in computer vision, and one for which the {\em  representation} of the database is a key issue.  We explore new ways for exploiting the structure of a database by representing it as a graph, and show how the rich information embedded in a graph can improve a bag-of-words-based location recognition method.  In particular, starting from a graph on a set of images based on visual connectivity, we propose a method for selecting a set of subgraphs and learning a local distance function for each using discriminative techniques.  For a query image, each database image is ranked according to these local distance functions in order to place the image in the right part of the graph.  In addition, we propose a probabilistic method for increasing the diversity of these ranked database images, again based on the structure of the image graph.  We demonstrate that our methods improve performance over standard bag-of-words methods on several existing location recognition datasets.
\end{itemize}

\subsection*{Learning to Match Images in Large-Scale Collections}
\begin{itemize}
\item February 2011 - November 2011
\item Many computer vision applications require computing structure and feature correspondence across a large, unorganized collection of images.  This is generally a difficult, computationally expensive process, because the set of matching image pairs is unknown in advance, and so good methods for quickly predicting which images match are critical. Image comparison method such as bag-of-words models or global features, are often used to predict similar pairs, but can be very noisy.  In this paper, we propose a new image matching approach that uses discriminative learning techniques---applied to training data gathered automatically during the image matching process---to gradually compute a better similarity measure for predicting whether two images overlap.  By using such a learned similarity measure, our algorithm can select image pairs that are more likely to match for performing further matching and geometric consistency checks, improving overall efficiency in the matching process.  Our approach processes a set of images in an iterative manner, alternately performing pairwise feature matching and learning an improved similarity measure. Our experiments show that our learned measures can significantly improve match prediction over both the standard {\em tf-idf} weighted similarity measure and more recent unsupervised techniques even with small amounts of training data, and can improve the overall speed of the image matching process by more than a factor of two.
\end{itemize}

%\subsection*{Graph Exploration Learning for Efficient 3D Reconstruction}
%\begin{itemize}
%\item September 2010 - January 2011
%\item An efficient structure from motion algorithm that incrementally reconstructs 3D models from large collection of images is implemented. First, pair-wise similarities of images are computed using visual words model. Then, we interleave image matching and model reconstruction, build- ing 3D model incrementally. During this process, to minimize bundle adjustment cost, an efficient algorithm defining strategies for choosing images to add is proposed, which makes use of predicted probabilities of successfully adding a candidate image. Online logistic regression is used to learn the function of computing such probabilities from descriptors obtained by applying PCA on original visual words features in high dimension. Results show that logistic regression combined with PCA is an effective approach in predicting probabilities and the graph exploration algorithm demonstrates good performance in both efficiency and accuracy.
%\end{itemize}

%\subsection*{Departure aircraft sequence optimization using EDA}
%\begin{itemize}
%\item April 2008 - May 2009
%\item How to minimize delays of flights in terminal area by optimization of departure aircraft sequence is a critical problem in the field of air traffic control (ATC). In this paper, estimation of distribution algorithm (EDA) with sliding window is implemented to solve this problem. Chromosomes that represents priority list are used in our method and fitness value is determined by average delay of aircrafts. By adding sliding window into traditional optimization process, it has solved the problem of oversized solution space caused by large number of aircrafts to be scheduled. Simulation results show that estimation of distribution algorithm has satisfactory overall performance in departure aircraft sequence optimization, and the sliding window parameters have significant effect on the optimization results.
%\item The efficient scheduling of airport runways is an important part of surface operations planning, with the goal of increasing the throughput of airports. The challenge mainly lies in optimizing different objective functions, while satisfying a variety of real world constraints. We implemented an aircraft scheduling and path planning simulation system utilizing several state-of-the-art algorithms to generate optimized aircraft schedule and path with conflict detection and resolution in real airport situations.
%\end{itemize}

%\subsection*{Multiple Motion Trajectory-Based Video Retrieval (Undergraduate Thesis)}
%\begin{itemize}
%\item February 2008 - June 2008
%\item A video retrieval system is built which extracts motion vectors from MPEG-1 video bit-stream and connects them to form motion trajectories. Then an improved reduction algorithm is proposed and utilized to obtain the most representative trajectories, which are then stored in database. When processing the query, the system uses a coarse-to-fine comparison strategy to reduce time cost. 2 types of query input are processed: Query by Example (QBE) and Query by Sketch (QBS). At last, all the algorithms are implemented and tested, and the result indicates the proposed reduction algorithm indeed excels the existing one and the system's performance is satisfactory.
%\end{itemize}

\subsection*{Approximate Computation of Curves on B-spline Surfaces}
\begin{itemize}
\item October 2006 - December 2007
\item Curves on surfaces play an important role in Computer Aided Geometric Design. Due to the considerably high degree of exact curves on surfaces, approximation algorithms are preferred in CAD systems. We present an algorithm to approximate the exact curve with a reasonably low degree curve which also lies completely on the B-spline surface. The Hausdorff distance between the approximate curve and the exact curve is controlled under the user-specified distance tolerance.
\end{itemize}

\section*{Teaching}
\begin{itemize}
\item CS 6670: Computer Vision (PhD-level course), Teaching Assistant, 2011 Spring, Cornell University.
\end{itemize}

\section*{Skills}
\begin{itemize}
\item Python (Proficient), C/C++ (Proficient), Java (Proficient), Unix Tools (Familiar), C\#(Familiar), MATLAB and Mathematica (Familiar) %, VHDL, Delphi
\item Familiar with image retrieval \& recognition algorithms, discriminative learning and graphical models
\item Familiar with applied statistics, stochastic processes, and optimization algorithms
\end{itemize}

\section*{Honors \& Awards}
\begin{itemize}
\item Excellent Graduate, Tsinghua University, China, 2008
\item Tsinghua Academic Excellence Scholarship, 1st Prize(Mitsubishi UFJ Scholarship), 2007
\item Tsinghua Comprehensive Excellence Scholarship, 2nd Prize (IBM Scholarship), 2007
\item Tsinghua Comprehensive Excellence Scholarship, 1st Prize (Toyota Scholarship), 2006
\item Tsinghua Comprehensive Excellence Scholarship, 1st Prize (POSCO Scholarship), 2005
\item 1st Prize of National College Physics Competition, 2005
%\item 2nd Prize of Chinese High School Physics Olympiad, 2003
%\item 3rd Prize of Chinese High School Chemistry Olympiad, 2003
\end{itemize}

%\section*{Leadership Experience}
%\begin{itemize}
%\item President, Cornell Chinese Tennis Club, July 2010 to June 2011.
%\item Class Leader, September 2007 to July 2008.
%\item President, Tennis Association of Tsinghua University, September 2006 to September 2007.
%\item Captain, Tsinghua Tennis Team, October 2006 to October 2007.
%\end{itemize}
%
%\section*{Volunteer Activities}
%\begin{itemize}
%\item "Information Service" Volunteer, China Open 2007, Sep. 2007
%\item "Campus Guide" Volunteer, Tsinghua University, Oct 2006
%\item Member, Volunteer Association of THSS (School of Software, Tsinghua University), Oct. 2006 to Jul. 2008
%\item Member, Communication and Study Department of THSS Student Union, Oct. 2004 to Jun. 2006
%\end{itemize}
%
%\section*{Other Extracurricular Activities}
%\begin{itemize}
%\item 3rd place, Men's Single of Ma Yuehan Tennis Cup, Tsinghua University, Apr. 2008
%%\item Tsinghua social practice research on the current social security system, Jan. 2006
%\item Half-marathoner in the ANA Beijing International Marathon, Oct. 2005
%\item 1st Prize, Teenager Painting and Calligraphy Exhibition, Hubei Province, China, Feb. 1998
%\item 2nd Prize, Golden Globe Awards of National Teenager Painting and Calligraphy Contest, China, Jan 1998
%\item ``Excellence'' in Level 5 in Professional Accordion Proficiency Test, Member of Musician Association of Hubei Province, China, Oct. 1993
%\end{itemize}
%
%\section*{Interests \& Hobbies}
%\begin{itemize}
%\item Unix, Python
%\item Tennis, Swimming
%\item Accordion Playing, Traditional Chinese Painting
%\end{itemize}

%\section*{Department Service}
%\begin{itemize}
%\item Faculty recruitment committee, 2010-11.
%\end{itemize}
%
%\section*{Academic Experience}
%
%\subsection*{North Carolina State University,
%  Department of Mathematics}
%
%\begin{itemize}
%
%\item Mathematics Tutor, Academic Support Program for Student Athletes,
%  Summer 2001.
%
%\item Research Assistant, Moody Chu and Robert Funderlic,
%  Spring 2003--Fall 2004.
%
%\end{itemize}
%
%\subsection*{Colorado School of Mines,
%  Department of Mathematical and Computer Sciences}
%
%\begin{itemize}
%
%\item Research Assistant, Willy Hereman, Summer 2002.
%
%\end{itemize}
%
%\subsection*{Duke University, Department of Economics}
%
%\begin{itemize}
%
%\item Research Assistant, Jacob Vigdor, Summer 2005--Summer 2006.
%
%\item Teaching Assistant, Ph.D. Microeconomic Analysis, Thomas Nechyba,
%  Fall 2005.
%
%\item Teaching Assistant, Ph.D. Game Theory, Curtis Taylor,
%  Spring 2006.
%
%\item Instructor, Microeconomics Qualifier Camp, Summer 2006.
%
%\item Research Assistant, Han Hong, Fall 2006--Spring 2007.
%
%\item Research Assistant, Aprajit Mahajan and Alessandro Tarozzi,
%  October--November 2006, March--June 2007.
%
%\item Research Assistant, Shakeeb Khan and Christopher Timmins,
%  March--June 2007.
%
%\item Teaching Assistant, Ph.D. Econometrics II, Shakeeb Khan,
%  Spring 2007.
%
%\item Research Assistant, Peter Arcidiacono, Patrick Bayer,
%  and Paul Ellickson, Fall 2007--Spring 2009.
%
%\item Research Assistant, Shakeeb Khan, Fall 2009--Spring 2010.
%
%\item Research Assistant, Aprajit Mahajan, July--September 2009.
%
%\end{itemize}
%
%\subsection*{The Ohio State University, Department of Economics}
%
%\begin{itemize}
%
%\item Assistant Professor, July 2010--Present.
%
%\end{itemize}
%
%
%\subsection*{Scientific Software}
%
%\begin{itemize}
%
%\item \href{http://jblevins.org/research/pdest}{PDESolutionTester},
%  A Mathematica program for the symbolic verification of exact solutions
%  of nonlinear partial differential equations,
%  with Jeff Heath and Willy Hereman (2002).
%
%\item \href{http://jblevins.org/research/dfbr}{DFBR},
%  A Stata program for distribution-free estimation of heteroskedastic
%  binary response models,
%  with Shakeeb Khan (2009).
%
%\end{itemize}
%
%\section*{Conference and Seminar Presentations}
%
%\begin{itemize}
%
%\item \href{http://jblevins.org/research/smcdmm}{Sequential Monte Carlo Methods for Estimating Dynamic Microeconomic Models}
%
%  \begin{itemize}
%
%  \item ERID Conference on Identification Issues in Economics,
%    Duke University, October 3, 2008.
%
%  \end{itemize}
%
%\item \href{http://jblevins.org/research/panel}{Partial Identification and Inference in Binary Choice and Duration
%    Panel Data Models}
%
%  \begin{itemize}
%
%  \item Triangle Econometrics Conference, Research Triangle Park, NC,
%    December 4, 2009.
%
%  \item University of Pittsburgh, Department of Economics,
%    January 15, 2010.
%
%  \item University of Illinois at Urbana-Champaign, Department of Economics,
%    February 16, 2010.
%
%  \item Annual Meetings of the Midwest Econometrics Group,  St. Louis, MO,
%    October 2, 2010.
%
%  \end{itemize}
%
%\item \href{http://jblevins.org/research/abbe}{Estimation of Dynamic Discrete Choice Models in Continuous Time}
%
%  \begin{itemize}
%
%  \item The Ohio State University, Department of Economics,
%    January 21, 2010.
%
%  \item University of Western Ontario, Department of Economics,
%    January 26, 2010.
%
%  \end{itemize}
%
%\end{itemize}
%
%\section*{Professional Activities}
%
%\begin{itemize}
%
%\item Member, Econometric Society, 2006--Present.
%
%\item Member, American Economic Association, 2010--Present.
%
%\item Referee for:
%
%  \begin{itemize}
%  \item \textit{Economic Inquiry} % 2008 (1)
%  \item \textit{Journal of Business and Economic Statistics} % 2009 (1)
%  \item \textit{Journal of Computational Finance} % 2010 (1)
%  \item \textit{Journal of Econometrics} % 2009 (1)
%  \item \textit{Scandinavian Journal of Statistics} % 2009 (1)
%  \end{itemize}
%
%\end{itemize}
%
%\section*{Department Service}
%
%\begin{itemize}
%\item Faculty recruitment committee, 2010-11.
%\end{itemize}
%
%\section*{Conferences and Workshops Attended}
%
%\begin{itemize}
%
%\item SIAM Conference on Applied Linear Algebra,
%  The College of William \& Mary,
%  July 15--19, 2003.
%
%\item Institute on Computational Economics,
%  Argonne National Laboratory,
%  July 17--21, 2006.
%
%\item North American Summer Meetings of the Econometric Society,
%  Duke University,
%  June 21--24, 2007.
%
%\item ERID Conference on Identification,
%  Duke University,
%  October 3--4, 2008.
%
%\item Cowles Foundation Conference on Applications of Structural Microeconomics,
%  Yale University,
%  June 24--25, 2009.
%
%\item Triangle Econometrics Conference,
%  Research Triangle Park, NC,
%  December 4, 2009.
%
%\item AEA Annual Meeting,
%  Atlanta, GA, January 3--5, 2010.
%
%\item Annual Meetings of the Midwest Econometrics Group,
%  St. Louis, MO, October 1--2, 2010.
%
%\end{itemize}
%
%
%\section*{Miscellaneous}
%
%\begin{itemize}
%
%  \item \textit{Security Clearance:} Special Sworn Status, U.S. Census Bureau.
%
%\end{itemize}

%\bigskip

% Footer
%\begin{center}
%  \begin{small}
%    Last updated: \today
%  \end{small}
%\end{center}

\end{document}
